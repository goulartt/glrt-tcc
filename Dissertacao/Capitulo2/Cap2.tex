%%%%%%%%%%%%%%%%%%%%%%%%%%%%%%%%%%%%%%%%
%---------- Segundo Capitulo ----------
\chapter{Fundamenta��o Te�rica}
\section{Revis�o Bibliogr�fica}
Existem alguns artigos descrevendo desafios de conseguir quebrar um mon�lito em microsservi�os, descrevendo suas dificuldades e cuidados a serem tomados. No Google Next '18 foi apresentado como que a Google Brit�nica trata sobre o assunto, apresentando poss�veis solu��es utilizando seus produtos, mas nenhum define um modelo a ser seguido e nem mostra um cen�rio real dessa refatora��o.

\chapter{Met�dos e Materiais}
\section{Ferramentas}
Ser� realizado um estudo de caso com objetivo de formalizar um modelo com as melhores solu��es se atingir a gl�ria da refatora��o de um sistema legado em uma arquitetura baseada em microsservi�os. Ser� utilizada como fontes artigos e conte�dos diversos existentes para formaliza��o do modelo e, por fim, ser� conduzido um experimento utilizando um software legado real de mercado como modelo de testes.

\section{Desenvolvimento}
O desenvolvimento consistir� em utilizar um software mon�lito real, feito em JAVA voltada ao cen�rio WEB, e conseguir refatorar esse software em uma arquitetura de microsservi�os. Por fim, ser� definido o modelo rafatora��o, independente da linguagem, baseado na experi�ncia real.
